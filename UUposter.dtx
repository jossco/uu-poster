% \iffalse meta-comment
%
% Copyright (C) 2011 by Joseph Scott <joseph.scott@it.uu.se>
% -------------------------------------------------------
% 
% This file may be distributed and/or modified under the
% conditions of the LaTeX Project Public License, either version 1.2
% of this license or (at your option) any later version.
% The latest version of this license is in:
%
%    http://www.latex-project.org/lppl.txt
%
% and version 1.2 or later is part of all distributions of LaTeX 
% version 1999/12/01 or later.
%
% \fi
%
% \iffalse
%<*driver>
\ProvidesFile{UUposter.dtx}
%</driver>
%<class>\NeedsTeXFormat{LaTeX2e}[1999/12/01]
%<class>\ProvidesClass{UUposter}
%<*class>
    [2011/09/19 v0.4b Class for making scientific posters in the style of Uppsala University]
%</class>
%
%<*driver>
\documentclass{ltxdoc}
\EnableCrossrefs         
\CodelineIndex
\RecordChanges
\begin{document}
  \DocInput{UUposter.dtx}
\end{document}
%</driver>
% \fi
%
% \CheckSum{0}
%
% \CharacterTable
%  {Upper-case    \A\B\C\D\E\F\G\H\I\J\K\L\M\N\O\P\Q\R\S\T\U\V\W\X\Y\Z
%   Lower-case    \a\b\c\d\e\f\g\h\i\j\k\l\m\n\o\p\q\r\s\t\u\v\w\x\y\z
%   Digits        \0\1\2\3\4\5\6\7\8\9
%   Exclamation   \!     Double quote  \"     Hash (number) \#
%   Dollar        \$     Percent       \%     Ampersand     \&
%   Acute accent  \'     Left paren    \(     Right paren   \)
%   Asterisk      \*     Plus          \+     Comma         \,
%   Minus         \-     Point         \.     Solidus       \/
%   Colon         \:     Semicolon     \;     Less than     \<
%   Equals        \=     Greater than  \>     Question mark \?
%   Commercial at \@     Left bracket  \[     Backslash     \\
%   Right bracket \]     Circumflex    \^     Underscore    \_
%   Grave accent  \`     Left brace    \{     Vertical bar  \|
%   Right brace   \}     Tilde         \~}
%
%
% \changes{0.1b}{2011-09-03}{Initial Version}
% \changes{0.2b}{2011-09-04}{xetex driver optional, changed image files}
% \changes{0.3b}{2011-09-05}{Ruled option now supports variable number of columns}
% \changes{0.4b}{2011-09-19}{fixed header height calculation in landscape mode}
%
% \GetFileInfo{UUposter.dtx}
%
% \DoNotIndex{\newcommand,\newenvironment}
% \DoNotIndex{\#,\$,\%,\&,\@,\\,\{,\},\^,\_,\~,\ }
% \DoNotIndex{\@ne}
% \DoNotIndex{\advance,\begingroup,\catcode,\closein}
% \DoNotIndex{\closeout,\day,\def,\edef,\else,\empty,\endgroup}
% 
%
% \title{The \textsf{UUposter} class\thanks{This document
%   corresponds to \textsf{UUposter}~\fileversion, dated \filedate.}}
% \author{Joseph Scott \\ \texttt{joseph.scott@it.uu.se}}
%
% \maketitle
%
% \section{Introduction}
%
% Put text here.
%
% \section{Usage}
%
% Put text here.
%
% \DescribeMacro{\dummyMacro}
% This macro does nothing.\index{doing nothing|usage} It is merely an
% example.  If this were a real macro, you would put a paragraph here
% describing what the macro is supposed to do, what its mandatory and
% optional arguments are, and so forth.
%
% \DescribeEnv{dummyEnv}
% This environment does nothing.  It is merely an example.
% If this were a real environment, you would put a paragraph here
% describing what the environment is supposed to do, what its
% mandatory and optional arguments are, and so forth.
%
% \StopEventually{\PrintChanges\PrintIndex}
%
% \section{Implementation}
%
% For simplicity, we'll derive everything from the standard |article|
% class.
%    \begin{macrocode}
\RequirePackage{xkeyval}

\newif\if@uuposter@xetex
\newif\if@uuposter@times
\@uuposter@timestrue
\newif\if@uuposter@helvet
\@uuposter@helvettrue
\newif\if@uuposter@mathtools
\@uuposter@mathtoolstrue
\newif\if@uuposter@red
\newif\if@uuposter@grey
\newif\if@uuposter@top
\newif\if@uuposter@landscape
\newif\if@uuposter@ruled
\newif\if@uuposter@sigil@corner
\newif\if@uuposter@sigil@margin
\newif\if@uuposter@sigil@left
\newif\if@uuposter@sigil@right
\newif\if@uuposter@sigil@top
\newif\if@uuposter@sigil@bottom
\newlength\logoboxwidth
\newlength\uuposter@TitleHeight
\newif\if@uuposter@TitleHeightSet
\DeclareOptionX{xetex}{
	\@uuposter@xetextrue
}
\DeclareOptionX{notimes}{
	\@uuposter@timefalse
}
\DeclareOptionX{nohelvet}{
	\@uuposter@helvetfalse
}
\DeclareOptionX{nomathtools}{
	\@uuposter@mathtoolsfalse
}
\DeclareOptionX{mathtools}[]{
	\PassOptionsToPackage{mathtools}{#1}
}
\DeclareOptionX{red}{
	\@uuposter@redtrue
	\@uuposter@greyfalse	
	\ExecuteOptionsX{logo=vit}
}
\DeclareOptionX{grey}{
	\@uuposter@redfalse
	\@uuposter@greytrue
	\ExecuteOptionsX{logo=4f}
}
\DeclareOptionX{top}{\@uuposter@toptrue}
\DeclareOptionX{landscape}{
	\ExecuteOptionsX{sidebarwidth=0.1,columns=3}
	\@uuposter@landscapetrue
}
\DeclareOptionX{portrait}{
	\ExecuteOptionsX{sidebarwidth=0.20,columns=2}
	\@uuposter@landscapefalse
}
\DeclareOptionX{columns}[2]{\def \uuposter@columns {#1}}
\DeclareOptionX{ruled}{\@uuposter@ruledtrue}

\DeclareOptionX{logo}[4f]{\def\uuposter@LogoInclude{\includegraphics[width=\logoboxwidth]{UU_logo_#1_84}}}

\newcommand{\uuposterSigil}{
}
\define@choicekey+{UUposter.cls}{sigil}[\val\nr]{NW,NE,SW,SE,margin}[SE]{
	\ifcase\nr\relax
		\renewcommand{\uuposterSigil}{\includegraphics[width=\sigilwidth]{UU_sigill_SO_10Svart}}
		\@uuposter@sigil@cornertrue
		\@uuposter@sigil@marginfalse
		\@uuposter@sigil@toptrue
		\@uuposter@sigil@lefttrue
		\@uuposter@sigil@bottomfalse
		\@uuposter@sigil@rightfalse
	\or 
		\renewcommand{\uuposterSigil}{\includegraphics[width=\sigilwidth]{UU_sigill_SV_10Svart}}
		\@uuposter@sigil@cornertrue
		\@uuposter@sigil@marginfalse
		\@uuposter@sigil@toptrue
		\@uuposter@sigil@righttrue
		\@uuposter@sigil@bottomfalse
		\@uuposter@sigil@leftfalse
	\or
		\renewcommand{\uuposterSigil}{\includegraphics[width=\sigilwidth]{UU_sigill_NO_10Svart}}
		\@uuposter@sigil@cornertrue
		\@uuposter@sigil@marginfalse
		\@uuposter@sigil@bottomtrue
		\@uuposter@sigil@lefttrue
		\@uuposter@sigil@topfalse
		\@uuposter@sigil@rightfalse
	\or
		\renewcommand{\uuposterSigil}{\includegraphics[width=\sigilwidth]{UU_sigill_NV_10Svart}}
		\@uuposter@sigil@cornertrue
		\@uuposter@sigil@marginfalse
		\@uuposter@sigil@bottomtrue
		\@uuposter@sigil@righttrue
		\@uuposter@sigil@topfalse
		\@uuposter@sigil@leftfalse
	\or
		\@uuposter@sigil@margintrue
		\@uuposter@sigil@cornerfalse
		\renewcommand{\uuposterSigil}{\includegraphics[width=\sigilwidth]{UU_marginalsigill_5Svart}}
	\fi
}{%
	\ClassWarning{UUposter}{Ignoring unrecognized value for option 'sigil' (valid options: NW,NE,SW,SE,margin)}
}
\DeclareOptionX{sigilwidth}[0.5]{
	\def\sigilproportion{#1}
}
\DeclareOptionX{sidebarwidth}[0.2]{
	\def\sidebarproportion{#1}
}
\DeclareOptionX{titleheight}[6em]{
	\@uuposter@TitleHeightSettrue
	\setlength\uuposter@TitleHeight{#1}
}

\DeclareOptionX{math}{\PassOptionsToPackage{\CurrentOption}{UUxetex}}
\DeclareOptionX{noxits}{\PassOptionsToPackage{\CurrentOption}{UUxetex}}
\DeclareOptionX{serif}{\PassOptionsToPackage{\CurrentOption}{UUxetex}}
\DeclareOptionX{sanserif}{\PassOptionsToPackage{\CurrentOption}{UUxetex}}
	

\DeclareOptionX*{\PassOptionsToClass{\CurrentOption}{a0poster}}
\ExecuteOptionsX{grey,sigilwidth,portrait}
\ProcessOptionsX\relax

\if@uuposter@landscape
	\PassOptionsToClass{landscape}{a0poster}
\else
	\PassOptionsToClass{portrait}{a0poster}
\fi
\LoadClass{a0poster}

\if@uuposter@mathtools
	\RequirePackage{mathtools}
\fi

\if@uuposter@xetex
	\RequirePackage{UUxetex}
\else
	\if@uuposter@times
		\RequirePackage{times}
	\fi
	\if@uuposter@helvet
		\RequirePackage{helvet}
	\fi
\fi

\RequirePackage{xcolor}
\definecolor{uuposterred}{RGB}{153,0,0}
\definecolor{uuposterlightgrey}{RGB}{230,230,230}
\definecolor{uupostermidgrey}{RGB}{190,190,190}
\definecolor{uuposterdarkgrey}{RGB}{130,130,130}

\if@uuposter@red
	\definecolor{uuposter@bg}{named}{uuposterred}
	\definecolor{uuposter@fg}{named}{white}
\else
	\if@uuposter@grey
		\definecolor{uuposter@bg}{named}{uuposterlightgrey}
		\definecolor{uuposter@fg}{named}{black}
	\fi
\fi
%    \end{macrocode}
% The footmisc package is used to fix footnotes in the sidebar minipage:
%    \begin{macrocode}
\RequirePackage[hang,norule,multiple]{footmisc}
\setlength{\footnotemargin}{2em}

\RequirePackage{graphicx}
%    \end{macrocode}
% Using the forloop package just for this seems ridiculous, so I don't:
%    \begin{macrocode}
\RequirePackage{ifthen}
\newcommand{\forloop}[5][1]{%
\setcounter{#4}{#2} %
\ifthenelse{ \value{#4}<#3 }%
{%
#5 %
\addtocounter{#4}{#1} %
\forloop[#1]{\value{#4}}{#3}{#4}{#5} %
}{%
#5 %
}}%

%    \end{macrocode}
% flowfram is used to control the poster layout:
%    \begin{macrocode}
\RequirePackage{flowfram}
\setlength{\vcolumnsep}{2\baselineskip}
\setlength{\columnsep}{2\baselineskip}
\newlength\sidebarwidth
\newlength\sidebaroffset
\setlength{\sidebarwidth}{\sidebarproportion\textwidth}
\setlength{\logoboxwidth}{\sidebarwidth}
\addtolength{\logoboxwidth}{-\vcolumnsep}
\newlength\logoboxoffset
\setlength{\logoboxoffset}{\vcolumnsep}
\if@uuposter@TitleHeightSet
	\if@uuposter@top
		\setlength{\logoboxwidth}{\uuposter@TitleHeight}
		\addtolength{\logoboxwidth}{-\vcolumnsep}
	\fi
\else
	\setlength\uuposter@TitleHeight{\logoboxwidth}
	\addtolength{\uuposter@TitleHeight}{-2\logoboxoffset}
	\@uuposter@TitleHeightSettrue
\fi
\if@uuposter@top
	\setlength{\sidebarwidth}{0pt}
	\setlength{\sidebaroffset}{0pt}
\else
	\setlength{\sidebaroffset}{\sidebarwidth}
	\addtolength{\sidebaroffset}{\vcolumnsep}
\fi

\newlength\sigilwidth
\if@uuposter@sigil@margin
	\if@uuposter@top
		\setlength{\sigilwidth}{\uuposter@TitleHeight}
	\else
		\setlength{\sigilwidth}{\sidebarwidth}
	\fi
\fi
\if@uuposter@sigil@corner
	\setlength{\sigilwidth}{\sigilproportion\textwidth}
	\newlength\sigiloffsetx
	\newlength\sigiloffsety
	\if@uuposter@sigil@right
		\setlength{\sigiloffsetx}{\textwidth}
		\addtolength{\sigiloffsetx}{-\sigilwidth}
	\else \if@uuposter@sigil@left
		\setlength{\sigiloffsetx}{\sidebarwidth}
	\fi \fi
	\if@uuposter@sigil@bottom
		\setlength{\sigiloffsety}{0pt}
	\else \if@uuposter@sigil@top
		\setlength{\sigiloffsety}{\textheight}
		\addtolength{\sigiloffsety}{-\sigilwidth}
	\fi \fi
	\newstaticframe{\sigilwidth}{\sigilwidth}{\sigiloffsetx}{\sigiloffsety}[background]
	\setstaticframe*{background}{valign=b}
\fi
\newlength\mainwidth
\setlength{\mainwidth}{\textwidth}
\addtolength{\mainwidth}{-\sidebaroffset}
\Ncolumntopinarea{static}{\uuposter@columns}{\uuposter@TitleHeight}{\mainwidth}{\textheight}{\sidebaroffset}{0pt}
\if@uuposter@sigil@corner
	\setstaticframe{2}{label={title}}
\else
	\setstaticframe{1}{label={title}}
\fi
\if@uuposter@ruled
	\newcounter{firstcol}
	\setcounter{firstcol}{1}
	\newcounter{secondcol}
	\forloop{2}{\uuposter@columns}{secondcol}{%
		\insertvrule{flow}{\thefirstcol}{flow}{\thesecondcol}
		\addtocounter{firstcol}{1}
	}
\fi

\newlength\uupostercolwidth
\getflowbounds{1}
\setlength{\uupostercolwidth}{\ffareawidth}

\newstaticframe{\sidebarwidth}{\textheight}{0pt}{0pt}[sidebar]
\newlength\authoroffset
\if@uuposter@top
	\setstaticframe*{title}{backcolor=uuposter@bg}
	\getstaticbounds*{title}
	\newstaticframe{\logoboxwidth}{\ffareaheight}{0pt}{\ffareay}[logobox]
	\setlength{\authoroffset}{\textwidth}
	\addtolength{\authoroffset}{-\logoboxwidth}
	\newstaticframe{\logoboxwidth}{\ffareaheight}{\authoroffset}{\ffareay}[authorbox]
	\addtolength{\authoroffset}{-\logoboxwidth}
	\newstaticframe{\authoroffset}{\ffareaheight}{\logoboxwidth}{\ffareay}[titlebox]
\else
	\setstaticframe*{sidebar}{backcolor=uuposter@bg}
	\if@uuposter@landscape
		\setlength{\authoroffset}{3em}
	\else
		\setlength{\authoroffset}{6em}
	\fi
\fi
\setallstaticframes{valign=t}
%    \end{macrocode}
% \begin{macro}{\uuposterTitleTextColor}
% \begin{macro}{\uuposterTitleTextStyle}
% Color and style of title text. Default color is UU red unless the top bar is colored,
% in which case the default foreground color is used instead.
%    \begin{macrocode}
\newcommand\uuposterTitleTextColor{%
	\if@uuposter@top
		\color{uuposter@fg}%
	\else
		\color{uuposterred}%
	\fi%
}
\newcommand\uuposterTitleTextStyle{%
	\sffamily\bfseries\veryHuge%
}
%    \end{macrocode}
% \end{macro}
% \end{macro}
%    \begin{macrocode}
\renewcommand\@maketitle{%
  \raggedright%
  \let \footnote \thanks
    {%
    \uuposterTitleTextColor%
	\uuposterTitleTextStyle%
    \@title \par}%
  \par
  }
\newcommand\uuposterBeforeTitle{}
\newcommand\uuposterTitleContent{\maketitle}
\newcommand\uuposterAfterTitle{}
\newcommand\uuposterTitle{%
	\uuposterBeforeTitle%
	\uuposterTitleContent%
	\uuposterAfterTitle}
%    \end{macrocode}
% \begin{macro}{\uuposterAuthor}
% Override this to set the author content in the document:
%    \begin{macrocode}
\newcommand\uuposterAuthor{}
%    \end{macrocode}
%\end{macro}
%    \begin{macrocode}
\newcommand{\uuposterAuthorBlock}{%
	\uuposterBeforeAuthor%
	\uuposterAuthorContent%
	\uuposterAfterAuthor%
}
\newcommand\uuposterAuthorFootSkip{%
	\if@uuposter@top
		\vspace{0.5\baselineskip}
	\else
		\vspace{\baselineskip}
	\fi%
}
\newlength\uuposterAuthorOffset
\if@uuposter@top
	\setlength{\uuposterAuthorOffset}{0pt}
\else
	\if@uuposter@landscape
		\setlength{\uuposterAuthorOffset}{3em}
	\else
		\setlength{\uuposterAuthorOffset}{6em}
	\fi
\fi
\newcommand\uuposterBeforeAuthor{%
	\if@uuposter@sigil@margin
		\vfill
	\else
		\vskip \uuposterAuthorOffset
	\fi%
}
\newcommand\uuposterAuthorTextColor{\color{uuposter@fg}}
\newcommand\uuposterAuthorTextStyle{\sffamily\Large}
\newcommand\uuposterAuthorContent{%
	\begin{minipage}{0.9\logoboxwidth}
		\renewcommand{\thempfootnote}{\textcolor{uuposter@fg}{\Large{{\arabic{mpfootnote}}}}}
		\uuposterAuthorTextColor\uuposterAuthorTextStyle\uuposterAuthor
		\uuposterAuthorFootSkip
	\end{minipage}%
}
\newcommand\uuposterAfterAuthor{}
\newcommand\uuposterInstTextStyle{\large\sffamily}
\newcommand\uuposterInstTextColor{\uuposterAuthorTextColor}
\newcommand{\uuposterInst}[1]{%
	\footnote{\uuposterInstTextColor\uuposterInstTextStyle{#1}}%
}

\newcommand\uuposterBeforeLogo{%
	\vspace{\logoboxoffset}
}
\newcommand\uuposterLogoContent{\uuposter@LogoInclude}
\newcommand\uuposterAfterLogo{}
\newcommand\uuposterLogo{%
	\uuposterBeforeLogo%
	\uuposterLogoContent%
	\uuposterAfterLogo%
}


\newcommand\uuposterSectionTextColor{\color{black}}
\newcommand\uuposterSectionTextStyle{\sffamily\Huge}
\newcommand\uuposterSubSectionTextColor{\uuposterSectionTextColor}
\newcommand\uuposterSubSectionTextStyle{\sffamily\huge}
\renewcommand\section{%
\@startsection{section}{1}{\z@}%
              {-3ex \@plus -1ex \@minus -.2ex}%
              {2.3ex \@plus.2ex}%
              {\uuposterSectionTextColor\uuposterSectionTextStyle}}
\renewcommand\subsection{%
\@startsection{subsection}{2}{\z@}%
              {-2.25ex\@plus -1ex \@minus -.2ex}%
              {1ex \@plus .2ex}%
              {\uuposterSubSectionTextColor\uuposterSubSectionTextStyle}}
\newcommand\uuposterBodyTextStyle{\normalsize}
\newcommand\uuposterBodyTextColor{\color{black}}        
\hyphenpenalty 10000
\tolerance 5000



%    \end{macrocode}
% Now we place the contents 
%    \begin{macrocode}
\AtBeginDocument{%
\if@uuposter@sigil@corner
	\begin{staticcontents*}{background}
		\raggedleft
		\uuposterSigil
	\end{staticcontents*}
\fi
\if@uuposter@top
	\begin{staticcontents*}{titlebox}%
		\uuposterTitle%
	\end{staticcontents*}
	\begin{staticcontents*}{logobox}
		\uuposterLogo
	\end{staticcontents*}
	\begin{staticcontents*}{authorbox}
		\uuposterAuthorBlock
	\end{staticcontents*}
\else
	\begin{staticcontents*}{title}%
		\uuposterTitle%
	\end{staticcontents*}
	\begin{staticcontents*}{sidebar}
	\begin{center}
	\uuposterLogo
	\uuposterAuthorBlock
	\end{center}
	\if@uuposter@sigil@margin
		\vfill
		\uuposterSigil
	\fi
	\end{staticcontents*}
\fi
\uuposterBodyTextColor\uuposterBodyTextStyle%
}

%    \end{macrocode}
%
% \begin{macro}{\dummyMacro}
% This is a dummy macro.  If it did anything, we'd describe its
% implementation here.
%    \begin{macrocode}
\newcommand{\dummyMacro}{}
%    \end{macrocode}
% \end{macro}
%
% \begin{environment}{dummyEnv}
% This is a dummy environment.  If it did anything, we'd describe its
% implementation here.
%    \begin{macrocode}
\newenvironment{dummyEnv}{%
}{%
%    \end{macrocode}
% \changes{v1.0a}{2004/11/05}{Added a spurious change log entry to
%   show what a change \emph{within} an environment definition looks
%   like.}
% Don't use |%| to introduce a code comment within a |macrocode|
% environment.  Instead, you should typeset all of your comments with
% \LaTeX---doing so gives much prettier results.  For comments within a
% macro/environment body, just do an |\end{macrocode}|, include some
% commentary, and do another |\begin{macrocode}|.  It's that simple.
%    \begin{macrocode}
}
%    \end{macrocode}
% \end{environment}
%
% \Finale
\endinput
